\section{Einleitung}
Das \ac{CoAP} ist ein für den Einsatz in Sensornetzen konzipiertes Transferprotokoll.
Spezielle Randbedingungen in solchen Netzen sind wenig \ac{RAM} und \ac{ROM} der einzelnen
Kommunikationsteilnehmer, welche z.B. durch 8-Bit Mikrocontroller realisiert werden.
Weiterhin bieten Sensornetze im Vergleich zu üblichen Computernetzen nur geringe Datenraten und
haben möglicherweise hohe Paketfehlerraten.

Ein mögliches Netz in dem \ac{CoAP} zum Einsatz kommen könnte, wäre
beispielsweise ein \ac{6LoWPAN}-Netz, in dem \ac{CoAP}-Nachrichten via \ac{UDP}
versendet werden.
Außer UDP in der Transportschicht, werden jedoch keine weiteren unterliegenden
Schichten für \ac{CoAP} spezifiziert.
Es wird lediglich versucht, den Overhead der einzelnen Nachrichten gering zu
halten.
Zudem hat \ac{CoAP} eine \ac{REST}-Architektur (siehe \cite{Fielding:2000:ASD:932295}), weshalb die Last auf
speicherarmen Knoten nicht so hoch ist, da keine Status gehalten werden müssen.
Wie genau das versucht wird zu erreichen, wird später in diesem Dokument
erläutert.
Die Entwicklung von \ac{CoAP} wird von der \ac{CoRE}-Gruppe vorangetrieben. Der
erste Entwurf wurde im Jahr 2010 veröffentlicht.
Zum Zeitpunkt der Erstellung dieses Dokuments war Version~13 (siehe \cite{draft-ietf-core-coap-13}) aktuell. Es wird
versucht auf einige Änderung einzugehen, die während der Entwicklung entstanden
sind.

\subsection{Features}
In \cite{draft-ietf-core-coap-13} wird \ac{CoAP} mit folgenden Features beworben:
\begin{itemize}
    \item Constrained web protocol fulfilling \ac{M2M} requirements.
    \item \ac{UDP} binding with optional reliability supporting unicast and multicast requests.
    \item Asynchronous message exchanges.
    \item Low header overhead and parsing complexity.
    \item \ac{URI} and Content-type support.
    \item Simple proxy and caching capabilities.
    \item A stateless \ac{HTTP} mapping, allowing proxies to be built providing access to \ac{CoAP}
    resources via \ac{HTTP} in a uniform way or for \ac{HTTP} simple interfaces to be realized
    alternatively over \ac{CoAP}.
    \item Security binding to \ac{DTLS}.
\end{itemize}

\subsection{Abgrenzung und Ziel}
Ziel von \ac{CoAP} ist es nicht, ein komprimiertes \ac{HTTP} umzusetzen, sondern viel mehr einen
Teil von \ac{REST}, wie es in \ac{HTTP} zum Einsatz kommt, bereitzustellen.
Insbesondere Optimierungen für eine \ac{M2M}-Kommunikation wie beispielsweise eingebaute Discovery-
Mechanismen, Unterstützung für Multicasts und asynchronen Nachrichtenaustausch sollen durch
\ac{CoAP} bereitgestellt werden.