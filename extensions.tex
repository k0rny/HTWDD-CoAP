\section{Erweiterungen}
\ac{CoAP} wie es in \cite{draft-ietf-core-coap-13} beschrieben ist,
bietet die Möglichkeit \ac{REST}ful-Dienste in einem Low-Power-Netzwerk zu integrieren.
Doch neben der Entwicklung des Protokolls selbst findet man auch viele Ideen auf
dem \ac{IETF}-Datatracker \footnote{\url{http://datatracker.ietf.org/wg/core/}}.
Einige Überlegungen betreffen die Integration von \ac{CoAP} in andere Netzwerke, wie
zum Beispiel: "`Transport of \ac{CoAP} over \ac{SMS}, \ac{USSD} and \ac{GPRS}"'
(siehe \cite{draft-becker-core-coap-sms-gprs-03}).
Andere Drafts befassen sich mit Sicherheitsaspekten
(siehe \cite{draft-bormann-core-ipsec-for-coap-00}), oder noch genauer mit der Thematik wie man
mit schlafenden Knoten in einem Sensornetz umgehen soll
(siehe \cite{draft-rahman-core-sleepy-01} und \cite{draft-rahman-core-sleepy-problem-statement-01}) und viele
weitere mehr.
Als Standard-Erweiterungen haben sich die drei folgenden herausgestellt.
\subsection{Block - nach \cite{draft-ietf-core-block-10}}
Für kleine Payloadgrößen muss man sich in \ac{CoAP} keine Gedanken über
Fragmentierung unterliegender Schichten machen. Hat man jedoch größere
Datenmengen zu übertragen, so wird einem durch Abbildung \ref{table:FramePayload}
schnell bewusst, dass auch beim Einsatz von \ac{CoAP} mit eventuell vielen Optionen nicht viel
Platz für den eigentlichen Payload bleibt. Sprengt man den 127-Byte-Rahmen, der vom
\ac{PHY} vorgegeben ist, so müssen die unteren Schichten eine
Fragmentierung durchführen. Um dies zu vermeiden, gibt es die Block-Erweiterung
in \ac{CoAP}. Hier wird die Fragmentierung auf die Applikationsschicht angehoben.
\begin{figure}[htbp]
    \centering
    \begin{minipage}{.61\textwidth}
    \lstinputlisting[basicstyle=\footnotesize]{sources/PacketPayload2.txt}
    \caption{Aufteilung des 802.15.4 Frames}
    \end{minipage}
    \label{table:FramePayload}
\end{figure}
Dies bringt den Vorteil, dass die unteren Schichten, sprich die Adaptions- oder \ac{IP}-schicht,
nicht damit belastet werden. Ein weiterer Vorteil ist das Acknowledgment der
einzelnen Pakete. Tritt ein Fehler auf, so muss nicht das gesamte Paket
wiederholt werden, sondern nur ein Teil davon.\\\\
Um den Blocktransfer zu beschreiben, werden 3 Attribute in der Block-Option
eingeführt.
\begin{description}
 \item[SZX] 
 Die im jeden Request oder Response übermittelte Paketgröße. Sechs verschiedene
 Größen sind vorgesehen, von $2^4=16$ bis $2^{10}=1024$~Bytes. Die Möglichkeit
 von sehr hohen Paketgrößen ist für den Einsatz von \ac{CoAP} mit anderen Schichten
 gedacht, als beispielsweise mit dem 802.15.4/\ac{6LoWPAN}/\ac{UDP} Stack. Bis auf den 
 letzten Block einer Blockübertragung muss der Payload genau diese Größe annehmen.
 Die Paketgröße wird in einem 3-Bit großen Feld angegeben, in der immer nur der 
 Zweierexponent abzüglich 4 eingetragen ist.
 \item[M-Flag] Dieses Attribut gibt an, ob es sich um das letzte Paket handelt, beziehungsweise
 ob noch mehr Pakete vorhanden sind. Die Information wird in einem Bit kodiert. 
\item[NUM] Die Blocknummer gibt an, welches Paket übermittelt, beziehungsweise angefragt,
 wurde. Sie beginnt bei 0 und kann bis zu 1048576 ($2^{20}$) laufen. Die
 Paketnummer wird entweder in 4,12 oder 20~Bit geschrieben.
\end{description}
\begin{figure}[htbp]
    \centering
    \begin{minipage}{.83\textwidth}
    \lstinputlisting[basicstyle=\footnotesize]{sources/BlockOption.txt}
    \caption{Optionen für den Blocktransfer}
    \end{minipage}
    \label{table:blockoption}
\end{figure}
Wie in Abbildung~\ref{table:blockoption} zu sehen ist, werden zwei Block-Optionen benötigt.
Die \verb!Block1!-Option wird verwendet, wenn der Client Daten im Block-Modus zu dem Server
senden möchte oder muss, weil der Server es verlangt.
Die \verb!Block2!-Option wird verwendet, wenn bei einer \verb!GET!-Anfrage an den Server die
Antwort im Block-Modus zurückgeschickt wird.
Die genaue Belegung der Attribute je nach Anfrage soll hier allerdings nicht
genauer erläutert werden.

Die Block-Option selbst hat keine Beschreibung, wieviele Pakete noch kommen. Es
gibt nur eine Paketnummer, die Blockgröße und ein Flag, was besagt, ob noch mehr
Blöcke vorhanden sind. Da es in manchen Anwendungen notwendig ist, wurde ebenso
auch die optionale Size-Option eingeführt. Mit Hilfe dieser Option kann man die
Größe des eigentlichen Payloads angeben.

Die Einführung der Block-Option hebt keinesfalls das \ac{REST}-Prinzip auf, denn
weiterhin beschreiben alle Anfragen sich soweit, dass sie vom Server ausgeführt
werden können, ohne sich einen Status zu merken.

\subsection{Observe - nach \cite{draft-ietf-core-observe-07}}
In Anbetracht der Tatsache, dass \ac{CoAP} dafür entwickelt ist, um mit möglichst wenig Overhead
einen \ac{REST}ful Dienst zu ermöglichen, treten Probleme bei der Anwendung von \ac{CoAP} in
Sensornetzen auf.
Dort möchte man beispielsweise recht schnell mitbekommen, wann sich eine Tür geöffnet hat.
Es bleibt also nur die Möglichkeit in engen Abständen den Zustand der Tür abzufragen.
Gehen wir ein Stück weiter und stellen uns die Tür zu einem sehr selten frequentierten Raum vor, so
wird sehr viel Aufwand und Netzlast erzeugt, um in einem aktzeptablen Abstand informiert zu werden,
dass sich der Status der Tür geändert hat.
Problem hier ist der Mechanismus der \ac{REST}-Struktur, in der immer der Client eine Anfrage
abschickt und eine Antwort erhält und die Kommunikation damit abgeschlossen ist.
\subsubsection{Überwachen von Ressourcen}
Mit der Idee der Überwachung wird ein neuer Mechanismus in \ac{CoAP} eingeführt.
Anstatt nur einmalig eine Antwort vom einen Server auf eine Anfrage zu bekommen, ist es möglich sich
von dem Server über die Repräsentation einer Ressource informieren zu lassen.
Das \ac{ODP}\cite{GAMMAETAL} wurde in \ac{CoAP} integriert mit der Observe-Option, welche in
Abbildung~\ref{table:observeoption} dargestellt ist.
 \begin{figure}[htbp]
    \centering
    \begin{minipage}{.94\textwidth}
    \lstinputlisting[basicstyle=\footnotesize]{sources/ObserveOption.txt}
    \caption{Die Observe-Option}
    \end{minipage}
    \label{table:observeoption}
\end{figure}
Alle \ac{REST}-Eigenschaften bleiben dabei erhalten, doch ist es nun einem Client möglich auf eine
Anfrage mehrere Antworten zu bekommen.
Um dies zu realisieren, gibt es die Möglichkeit sich bei dem Server als Observer in eine Liste
eintragen zu lassen.
Ist man in der Liste, so erhält man von nun an bei Statuswechseln, gewissen abgelaufenen
Zeitabständen eine Notification über die neue/aktuelle Repräsentation der überwachten Ressource.
Die genaue Implementation soll in diesem Dokument hier nicht dargestellt werden, doch sei darauf
hingewiesen, das es ebenso möglich ist die Observe-Option mit der ETag-Option zu kombinieren.
\subsubsection{Probleme mit Safe-to-Forward}
Seit dem 6. Februar 2013 ist eine Diskussion in der Mailingliste zu beobachten, bei der es darum
geht, dass es Probleme seit dem Draft-12 (siehe \cite{draft-ietf-core-coap-12}) von \ac{CoAP} gibt.
In diesem Draft wurde die Unterscheidung in Un-/Safe-to-Forward eingeführt.
Diese Unterscheidung führt nun dazu, dass Proxys, welche zwar die Observe Funktion verstehen, aber
möglicherweise andere Optionen nicht, welche die Überwachung aber beeinflussen.
Ebenso tritt es dadurch auf das Clienten sich gegenseitig aus der Liste der Observer löschen.
Derzeit tendiert die Diskussion wohl dahin ein weitere Unterscheidung der Optionen einzubauen.
Neben dem Cache-Key, gibt es womöglich demnächst ein Observe-Key, welcher genutzt werden soll um
die oben angesprochenen Probleme zu beheben.
\subsection{Groupcomm nach \cite{draft-ietf-core-groupcomm-05}}
\subsubsection{Notwendigkeit}
Für \ac{CoAP} selbst wird nicht geklärt, wie eine Gruppenkommunikation auszusehen hat.
Eine Gruppenkommunikation ist jedoch in der Praxis unerlässlich.
Ein Beispiel hierfür wäre die Einteilung von Sensoren/Aktoren in einem Netz nach Typ oder Standort.
Eine mögliche Gruppe von Aktoren wären alle Lichtschalter im Haus oder alle vorhandenen
Thermostate.
Es wird Wert darauf gelegt, dass sowohl in \ac{CoAP} als auch in \ac{IP}-Multicasts keine
Änderungen hierfür vorgenommen werden müssen.
Insbesondere wie einzelne Protokolle für verschiedene Szenarien miteinander eingesetzt werden
müssen wird in dem Dokument erläutert.
\subsubsection{Richtlinien}
Da \ac{IP}-Multicasts weit verbreitet sind, bildet dieser Mechanismus die Grundlage für die
Gruppenkommunikation.
Um Multicasts über mehrere Subnetze zu ermöglichen ist es nötig Routing-Protokolle oder
Forwarding-Protokolle einzusetzen.
Für den Einsatz denkbar wäre hier beispielsweise \ac{RPL} oder \ac{MPL}.
Weiterhin wird noch ein sogenanntes Listener-Protokoll benötigt, um einzelne Geräte Gruppen
zuzuordnen.\\\\
Die Gruppen-\ac{URI} muss in der Nachricht die Request-\ac{URI} sein.
Der Authority-Teil muss eine Gruppen-\ac{IP}-Multicast-Adresse oder ein Hostname sein, welcher zu einer
solchen aufgelöst werden kann.
Folgende Beispiele für mögliche Gruppen-\ac{URI}s werden aufgelistet:
\begin{description}
\item[all.bldg6.example.com] all nodes in building 6
\item[all.west.bldg6.example.com] all nodes in west wing, building 6
\item[all.floor1.west.bldg6.examp...] all nodes in floor 1, west wing, building 6
\item[all.bu036.floor1.west.bldg6...] all nodes in office bu036, floor1, west wing, building 6
\end{description}
Der Gruppen-\ac{FQDN} muss via \ac{DNS} aufgelöst werden können.\\\\
Weiterhin wird geklärt, wie ein Knoten die Mitgliedschaft einer Multicast-Gruppe ankündigen kann.
Es wird die Vorgehensweise für die Protokolle \ac{MLD}, \ac{RPL}, \ac{MPL} kurz erläutert, was im
Umfang dieses Dokuments hier aber zu weit führen würde.\\\\
Für Multi-LoWPAN-Szenarien wird empfohlen, Filtering-Regeln einzuführen, um unnötige Netzlast zu
vermeiden. 
\paragraph{Filtern durch \ac{6LBR} mit Hilfe von Routinginformationen}
Dies ermöglicht es dem \ac{6LBR}, Multicast-Nachrichten nur in Netze weiterzuleiten, in denen auch
ein Empfänger für die Nachricht existiert
\paragraph{Filtern durch \ac{6LBR} mit Hilfe von \ac{MLD}-Reports}
Gleich dem vorherigen Verfahren, jedoch basierend auf \ac{MLD}-Reports.
\paragraph{Filtern durch \ac{6LBR} mit Hilfe von Einstellungen}
Nutzen von Filter-Tabellen mit Blacklists bzw. Whitelists für alle \ac{6LBR} bzw. für spezifische
\ac{6LBR}.
\subsubsection{Probleme}
Nach draft-13 (siehe \cite{draft-ietf-core-coap-13}) von \ac{CoAP} muss die Gruppenkommunikation im
\ac{NoSec}-Modus erfolgen.
Weiterhin darf kein \ac{DTLS}-gesichertes \ac{CoAP} und auch kein \ac{IPSec} verwendet werden.
Aus diesen Gründen müssen Sicherheitsvorkehrungen auf anderen Schichten vorgenommen werden.
Für die Zukunft wird darauf spekuliert, dass \ac{DTLS}-basierte \ac{IP}-Multicasts oder andere
Herangehensweisen die Sicherheitsprobleme lösen könnten.
Aktuell ist aber noch kein wirklicher Lösungsansatz für diese Probleme vorhanden.